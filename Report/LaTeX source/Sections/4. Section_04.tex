\section{Results and Discussion}
\label{sec:Results-and-Discussion}
As mentioned earlier, this section serves to present a graphical illustration of the results relevant to the experiments outlined in Section \ref{sec:Methodology} alongside a brief discussion.

\subsection{Experiment 1}
\label{subsec:Results-and-Discussion:Experiment-1}
The first of our experiments explored the relationship between the number of cars on a square lattice. Notably, we examined whether the BML traffic model would achieve a free-flowing state of traffic given that the total number of cars $(n)$ did not exceed the length of one of the sides of the aforementioned square lattice $\left(n \leq \frac{X}{2}\right)$. To ascertain this, we ran simulations using our implementation of the BML traffic model under the set of hyperparameters listed in Section \ref{subsec:Methodology:Experiment-1} and averaged the results over a total of 10 runs. Figure \ref{fig:Experiment-1.1} denotes the (averaged) mean velocity for each simulation under varying values of $\rho$. We note that for values of $n$ where $n \sim \frac{N}{2}$, the system tends to consistently reach a velocity of 1 and a free-flowing state of traffic which is in line with the theory outlined by \citeauthor{Austin} in their paper. In accordance with their paper, we observe that, for values of  $n < \frac{1}{2}(N)$, regardless how the cars are distributed in our $N \times N$ lattice then there will always exist an empty arc (over the diagonals of the square lattice in question) of length at least 2, and so the system will self-organize to a velocity of 1 in a finite amount of iterations. In the interest of preserving time and preventing this paper from being overtly lengthy, we direct the reader to the original paper, located \href{https://arxiv.org/pdf/math/0607759.pdf}{here}, to better understand how \citeauthor{Austin} came about to prove this proposition. 

\begin{figure}[H]
    \centering
    \includegraphics[width=\linewidth]{Images/Section 4/Experiment 1/1.2.pdf}
    \caption{Mean velocity of the simulations run under the set of hyperparameters pertaining to experiment 1. A mean velocity of $1.0$ denotes that the system experienced a smooth flow of traffic straight from the get-go and (in the worst case) experienced a number of collisions around the start.}
    \label{fig:Experiment-1.1}
\end{figure}

\subsection{Experiment 2}
\label{subsec:Results-and-Discussion:Experiment-2}
The results of experiment 2 are meant to verify that, as $\rho$ approaches 1, the BML traffic model will reach a globally jammed phase, in which no car can make a move, infinitely often. To do this, we ran we ran simulations using our implementation of the BML traffic model under the set of hyperparameters listed in Section \ref{subsec:Methodology:Experiment-2}, and once again averaged the results over a total of 10 runs per value of $\rho$. Our findings are outlined in Figures \ref{fig:Experiment-2.1}, \ref{fig:Experiment-2.2}, \ref{fig:Experiment-2.3}, and \ref{fig:Experiment-2.4}. The basis of the arguments formulated by \citeauthor{Omer} revolve around the existence of blocking paths in configurations where $\rho > \rho_c$. To best understand this, take an initialization in which $\rho = 1:$  each and every car is immediately blocked by a car positioned directly in front of it and in this way no cars can make a move due to an infinite chain of blocking cars that define a set of \textit{"blocking paths"}. While this logic does not immediately extend to configurations in which $\rho < 1$, as such a chain will always be broken by an empty space, if we consider the previously defined network of blocking paths for a configuration at $\rho = 1$, then it is likely that these blocking paths still exist in systems configured at values of $\rho < 1$ (considering the fact that taking $\rho < 1$ is akin to removing a proportion of cars from a $\rho = 1$ configuration.) See Figure \ref{fig:Blocking-Paths-1} for an illustration.

\begin{figure}[H]
    \centering
    \includegraphics[width=0.475\linewidth]{Images/Section 4/Blocking-Paths-1.pdf}
    \caption{An illustration of blocking paths as per the findings of \citeauthor{Omer}.  Image source: \cite{Omer} licensed under \href{https://creativecommons.org/licenses/by/3.0/}{CC BY 3.0}.}
    \label{fig:Blocking-Paths-1}
\end{figure}

\begin{figure}[H]
    \centering
    \includegraphics[width=\linewidth]{Images/Section 4/Experiment 2/2.1.pdf}
    \caption{Mean velocity per simulation averaged over a total of 10 runs per value of $\rho$.}
    \label{fig:Experiment-2.1}
\end{figure}

\vspace{3em}

\begin{figure}[H]
    \centering
    \includegraphics[width=\linewidth]{Images/Section 4/Experiment 2/2.2.pdf}
    \caption{Percentage of cars that moved at the last time step recorded for each of our simulations averaged over a total of 10 runs per value of $\rho$.}
    \label{fig:Experiment-2.2}
\end{figure}

\vspace{3em}

\begin{figure}[H]
    \centering
    \includegraphics[width=\linewidth]{Images/Section 4/Experiment 2/2.3.pdf}
    \caption{Time step that the system jammed (if applicable) for each of our simulations averaged over a total of 10 runs per value of $\rho$.}
    \label{fig:Experiment-2.3}
\end{figure}

\vfill\null

\begin{figure}[H]
    \centering
    \includegraphics[width=\linewidth]{Images/Section 4/Experiment 2/2.4.pdf}
    \caption{A deeper look at the velocity per iteration of 3 different systems at 3 different densities that exhibit the different states of traffic flow associated with the BML traffic model.}
    \label{fig:Experiment-2.4}
\end{figure}

\subsection{Experiment 3}
\label{subsec:Results-and-Discussion:Experiment-3}
The final experiment pertains to the emergence of periodic (with respect to time) intermediate states in the BML traffic model that capture the essence of both the jammed state as well as the free-flowing state. In particular, the findings of \citeauthor{DSouza}, show that for a lattice of coprime dimensions and for values of $\rho \sim \rho_c$ the system will self-organize into periodic arrangements that consists of both jams as well as smoothly flowing traffic. To attempt to recreate this we initialized a $144 \times 89$ rectangular lattice with a density of $38\%$ and ran the simulation for a total of $7,500$ iterations. We noted the velocity per time step of the simulation and found that at around $t \sim 4,000$ the system self-organized to the previously defined intermediate state. The findings of our results can be seen in Figures \ref{fig:Experiment-3.1}, \ref{fig:Experiment-3.2}, and \ref{fig:Experiment-3.3}.

\begin{figure}[H]
    \centering
    \includegraphics[width=\linewidth]{Images/Section 4/Experiment 3/Velocity.pdf}
    \caption{Velocity per time step of a simulation run on a rectangular lattice with coprime dimensions over a total of $7,500$ time steps.}
    \label{fig:Experiment-3.1}
\end{figure}

\vfill\null

\begin{figure}[H]
    \centering
    \includegraphics[width=\linewidth]{Images/Section 4/Experiment 3/Velocity_zoomed.pdf}
    \caption{A closer look at the velocity of the system at iterations $6,000 \rightarrow 7,500$.}
    \label{fig:Experiment-3.2}
\end{figure}

\begin{figure}[H]
    \centering
    \includegraphics[width=\linewidth]{Images/Section 4/Experiment 3/Autocorrelation.pdf}
    \caption{Plotting the autocorrelation of the velocity of the system in an attempt to illustrate the periodic nature of the intermediate state that the system self-organized to.}
    \label{fig:Experiment-3.3}
\end{figure}


\begin{figure}[H]
        \myfloatalign
        \subfloat[Time step: 6250.]
        {
                \includegraphics[width=0.45\linewidth]{Images/Section 4/Experiment 3/Simulation_step_6250.pdf}
        } \quad
        \subfloat[Time step: 6415.]
        {
                \includegraphics[width=0.45\linewidth]{Images/Section 4/Experiment 3/Simulation_step_6415.pdf}
        } \quad
        \caption{A snapshot of our system at 2 separate iterations (165 iterations apart) illustrating the intermediate phase characterized by arrangements of traffic jams as well as smoothly flowing traffic.}
\end{figure}