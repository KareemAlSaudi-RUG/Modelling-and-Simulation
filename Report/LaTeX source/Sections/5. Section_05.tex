\section{Conclusion}
\label{sec:Conclusion}
To conclude, in this paper we recreated major findings with respect to the BML traffic model by means of a Python implementation of the model and simulation runs over a variable set of hyperparameters that pertained to the aforementioned findings. In doing this, we explored the relationship of the traffic density and state of traffic flow and, with respect to the BML traffic model, we were able to better understand how simple changes with respect to initial configurations of the system can have drastic effects on whether or not the model would ultimately self-organize to either extreme with regards to traffic flow, be that a free-flowing state of traffic or a globally jammed state. Given the computational limitations and time constraints at the time of writing this project, the number of runs per experiment were limited alongside the overall size of the lattice(s) we explored which, given the nature of the project, did not come with any major ramifications although it would have been interesting to explore a wider array of initial configurations and even finer increments with respect to $\rho$. Exploring variations of the model,  \ie on a k-dimensional lattice $(k \geq 3)$, or the \textit{Klein bottle} variation or either of the non-deterministic, randomized variation or the open-boundary (chain) variation would also have been interesting and can be considered for future work. Other cellular automaton models, such as the Nagel-Schreckenberg model that bring in the concepts of car velocity and acceleration, would also be interesting to look at for future work.