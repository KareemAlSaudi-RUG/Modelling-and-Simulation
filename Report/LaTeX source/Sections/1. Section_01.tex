\IEEEraisesectionheading
{
\section{Introduction}
\label{sec:Introduction}
}
\IEEEPARstart{A}s our reliance on automobiles for travelling continues to grow so too does the increase in the congestion that occurs on our freeways. This congestion, oftentimes referred to as a \textit{traffic jam}, is a widespread phenomenon that occurs for any one of a number of reasons including, but not limited to, an increase in traffic demand that exceeds the capacity of the freeway, a reduction in freeway operation speed as a result of geometric constraints with respect to the layout of the roads of said freeway itself or the occurrence of events that are not entirely predictable such as a traffic accident or adverse weather conditions. Regardless of the cause, the resulting economic losses as a result of delays in traffic are enormous and are cause for concern for major governing bodies \cite{DYin}. Given the increased attention that problems with traffic have accrued; various approaches have been applied throughout the years in order to describe the collective properties of traffic flow \cite{Nagatani} with the intent to optimize and improve on what has become an urgently deteriorating situation \cite{Helbing}. Of these approaches is a self-organizing cellular automaton traffic flow model called the \gls{bml} traffic model. Throughout the scope of this project, and subsequently this report, we will be exploring how the \gls{bml} traffic model simulates traffic flow and traffic jams by means of implementing the model from scratch and performing experiments over a variable set of hyperparameters. The following sections of the paper will be organized as follows: Section \ref{sec:The-BML-Traffic-Model} serves to provide a brief overview of the history of the \gls{bml} traffic model while Section \ref{sec:Methodology} serves to outline the details of our experiments. Finally, Sections \ref{sec:Results-and-Discussion} and \ref{sec:Conclusion} serve to present the results of our experiments alongside a short discussion and concluding thoughts.